\section{Casos de Uso}\label{casosdeuso}
\par 
A seguir, detalha-se cinco das mais importantes funcionalidades do sistema. É apresentado os casos de usos de interação do ator principal, o Funcionário, e os fluxos principais e alternativos. Os demais casos de usos se encontram na seção de diagramas:

%CDU1
\par
\textbf{Nome:} Autenticar Funcionário. 
\par
\textbf{Descrição:} Autenticação dos funcionários para uso do sistema.
\par 
\textbf{Identificador:} CDU01.
\par
\textbf{Ator Primário:} Funcionário.	
\par
\par
\textbf{Fluxo principal}\par
\begin{tabular}{|p{7cm}|p{7cm}|}
	\hline 
	Funcionário & Sistema \\ 
	\hline 
	1 - Inserir cpf e senha  &  \\ 
	\hline 
	& 
	
	2 - Valida dados inseridos 
	\\ 
	\hline 
	& 
	
	3 - Exibe opções disponíveis
	\\ 
	\hline 
\end{tabular} 
\vspace{12px}
\par
\textbf{Fluxo Alternativo}(Login ou senha incorreto, funcionário inexistente ou inativado)\par
\begin{tabular}{|p{7cm}|p{7cm}|}
	\hline 
	Funcionário & Sistema \\ 
	\hline 
	1 - Inserir cpf e senha  &  \\ 
	\hline 
	& 
	
	2 - Valida dados inseridos 
	\\ 
	\hline 
	& 
	
	3 - Exibe mensagem de falha		
	\\ 
	\hline 
\end{tabular} 
\vspace{12px}

%CDU2
\par
\textbf{Nome:} Cadastrar Cliente.
\par
\textbf{Descrição:} Funcionário cadastra dados do cliente no sistema.
\par 
\textbf{Identificador:} CDU02.
\par
\textbf{Ator Primário:} Funcionário	
\par
\textbf{Precondição:} Funcionário deve estar autenticado no sistema.
\par
\par
\textbf{Fluxo principal}\par
\begin{tabular}{|p{7cm}|p{7cm}|}
	\hline 
	Funcionário & Sistema \\ 
	\hline 	
	1 - Selecionar opção de cadastrar cliente &  \\ 
	\hline 
	& 
	
	2 - Exibir formulário de cadastro 
	\\ 
	\hline 
	3 - Preencher dados de cadastro
	& 		
	
	\\ 
	\hline 
	4 - Selecionar opção de confirmar cadastro.
	& 
	
	\\ 
	\hline 
	& 	
	
	5 - Salvar cadastro. 	
	\\ 
	\hline 
	& 
	
	6 - Exibir Mensagem “Cadastro Realizado”.
	\\ 		
	\hline 
\end{tabular} 
\vspace{12px}
\par
\textbf{Fluxo Alternativo}(Cliente já cadastrado )\par
\begin{tabular}{|p{7cm}|p{7cm}|}
	\hline 
	Funcionário & Sistema \\ 
	\hline 	
	1 - Selecionar opção de cadastrar cliente &  \\ 
	\hline 
	& 
	
	2 - Exibir formulário de cadastro 
	\\ 
	\hline 
	3 - Preencher dados de cadastro
	& 		
	
	\\ 
	\hline 
	4 - Selecionar opção de confirmar cadastro.
	& 
	
	\\ 
	\hline 
	& 	
	
	5 - Retornar Mensagem “Usuário já cadastrado.”	
	\\ 
	\hline 
	& 
	
	6 - Exibir opção de atualizar dados  ou Sair.
	\\ 		
	\hline 
\end{tabular}
\vspace{12px}

%CDU3
\par
\textbf{Nome:} Orçar Serviços.
\par
\textbf{Descrição:} Gerar orçamento de um serviço. 
\par 
\textbf{Identificador:} CDU03
\par
\textbf{Ator Primário:} Funcionário	
\par
\textbf{Precondição:} Funcionário deve estar autenticado no sistema.
\par	
\par
\textbf{Fluxo principal}\par
\begin{tabular}{|p{7cm}|p{7cm}|}
	\hline 
	Funcionário & Sistema \\ 
	\hline 
	
	
	1 - Selecionar opção “Serviços”. 
	&  \\ 
	\hline 
	& 
	
	2 - Exibir tela menu de Serviços.
	\\ 
	\hline 
	
	
	3 - Selecionar a opção “Orçar Serviço”
	&  \\ 
	\hline 
	& 
	
	4 - Exibir lista de serviços cadastrados.
	\\ 
	\hline 
	
	
	5 - Selecionar um ou mais serviços.
	&  \\ 
	\hline 
	
	
	6 - Selecionar opção “Continuar”
	&  \\ 
	\hline 
	& 
	
	7 - Exibir tela com preços para cada serviço com opção de alterar para serviços sem preço fixo. 
	\\ 
	\hline 
	
	
	8 - Alterar os preços de serviço que possuam opção de alterar preço.
	&  \\ 
	\hline 
	
	
	9 - Selecionar opção “Adicionar peça”
	&  \\ 
	\hline 
	& 
	
	10 - Exibir tela para seleção de peças. 
	\\ 
	\hline 
	
	
	11 - Selecionar uma ou mais peças.
	&  \\ 
	\hline 
	
	
	12 - Selecionar opção “Continuar”.
	&  \\ 
	\hline 
	& 
	
	13 - Exibir tela de resumo com serviço(s) e peça(s) selecionados .
	\\ 
	\hline 
	
	
	14 - Inserir placa do veículo do cliente. 
	&  \\ 
	\hline 
	
	
	15 - Selecionar opção “Confirmar orçamento”.
	&  \\ 
	\hline 
	& 
	
	16 - Exibir Documento de Orçamento com descrição dos serviços, placa do veículo, código de identificação e  custo total.
	\\ 
	\hline 
	
	
	17 - Selecionar opção “imprimir”.
	&  \\ 
	\hline 
	& 
	
	18 - Encaminhar documento para impressão.
	\\ 
	\hline 
\end{tabular} 
\vspace{12px}	

%CDU4
\par
\textbf{Nome:} Vender Veículo 
\par
\textbf{Descrição:} Realizar venda de veículo 
\par 
\textbf{Identificador:} CDU04
\par
\textbf{Ator Primário:} Funcionário	
\par
\textbf{Precondição:} Funcionário deve estar autenticado no sistema.
\par	
\par
\textbf{Fluxo principal}\par
\begin{tabular}{|p{7cm}|p{7cm}|}
	\hline 
	
	
	Funcionário
	& 
	
	Sistema
	\\ 
	\hline 
	
	
	1 - Selecionar opção de veículos.
	&  \\ 
	\hline 
	& 
	
	2 - Exibir tela menu de veículos.
	\\ 
	\hline 
	
	
	3 - Selecionar opção de “Vender Veículo”
	&  \\ 
	\hline 
	& 
	
	4 - Exibir veículos para venda.
	\\ 
	\hline 
	
	
	5 - Selecionar veículo para venda.
	&  \\ 
	\hline 
	
	
	6 - Selecionar opção “Continuar.
	&  \\ 
	\hline 
	& 
	
	7 - Exibir tela para seleção de cliente
	\\ 
	\hline 
	
	
	8 - Seleciona cliente
	&  \\ 
	\hline 
	
	
	9 - Seleciona opção “Continuar”. 
	&  \\ 
	\hline 
	& 
	
	10 - Atualiza Quantidade do Veículo no Estoque
	\\ 
	\hline 
	& 
	
	11 - Exibe documento de comprovação de venda. 
	\\ 
	\hline 
	& 
	
	12 - Gera cobrança.
	\\ 
	\hline 
	
	
	13 - Seleciona opção “imprimir comprovante de venda”.
	&  \\ 
	\hline 
	
	
	14 - Seleciona opção “imprimir cobrança”.
	&  \\ 
	\hline 
\end{tabular}  
\vspace{12px}


%CDU5
\par
\textbf{Nome:} Autenticar Funcionário.
\par
\textbf{Descrição:}  Atualizar quantidade de peças no estoque.	
\par 
\textbf{Identificador:} CDU05
\par
\textbf{Ator Primário:} Funcionário.	
\par
\textbf{Precondição:} Funcionário deve estar autenticado no sistema.
\par
\par
\textbf{Fluxo principal}\par
\begin{tabular}{|p{7cm}|p{7cm}|}
	\hline 
	
	
	Funcionário
	& 
	
	Funcionário
	\\ 
	\hline 
	
	
	1 - Selecionar opção de peças.
	&  \\ 
	\hline 
	& 
	
	2 - Exibir menu de peças
	\\ 
	\hline 
	
	
	3 - Selecionar opção de gerenciar
	&  \\ 
	\hline 
	& 
	
	4 - Exibir listagem de peças.
	\\ 
	\hline 
	
	
	5 - Clicar em uma Peça.
	&  \\ 
	\hline 
	
	
	6 - Atualizar Quantidade de Peças.
	&  \\ 
	\hline 
	
	
	7 - Confirmar alteração.
	&  \\ 
	\hline 
	& 
	
	8 - Salvar alterações.
	\\ 
	\hline 
\end{tabular} 
\vspace{12px}
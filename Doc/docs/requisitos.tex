\section{Levantamento de Requisitos}\label{requisitos}

\subsection{Propósito do Documento}
\par
O objetivo deste documento é detalhar a descrição de requisitos do software OpenCarShop, deixar claro a motivação do desenvolvimento do sistema, bem como funcionalidades, interfaces, componentes, interações e restrições que o software contém. Este documento, deve ser aprovado pelos stakeholders, e assim, servir de referência para o time de desenvolvimento, auxiliando na evolução do software.




\subsection{Escopo do Produto}
\par
O OpenCarShop é um sistema de gestão que controlará os setores de venda de veículos, estoque, realização de orçamentos de serviços de uma concessionária de veículos de única marca. 
\par
Uma base de dados de veiculos, pecas, servicos, clientes e funcionários deve ser produzida e atualizada a medida que os usuários do sistema, os funcionários, alterem e adicionem tais dados durante a utilização do sistema.
\par
Os funcionários que irão interagir com o software o farão através de seus desktops. O software necessita de conexão com o servidor de dados para que os funcionários se autentiquem no sistema e manipularem os dados. 




\subsection{Definições e Abreviações}
	\subsubsection{Definições}
	
	\begin{itemize}
		\item Funcionário: Ator principal do sistema.
		\item Orçamento: Levantamento de preços  de serviços e peças atreladas a esses serviços.
		\item Peça: Peça mecânica ou acessório veicular.
		\item Serviço: Serviço veicular, reparo, manutenção, instalação de peça.
	\end{itemize}
	
	\subsubsection{Abreviações}
	\begin{itemize}
	\item RF: Requisito Funcional.
	\item RNF: Requisito Não Funcional.
	\item CDU: Caso de Uso.
	\end{itemize}



\subsection{Referências}

	\begin{itemize}
		\item[1]  Material usado nas aulas da disciplina Desenvolvimento de Software II ministrada pelo professor Michel dos Santos Soares disponibilizado em www.sigaa.ufs.br 
		\item[2] Pressman, Roger. Engenharia de Software: Uma abordagem profissional. Porta Alegre: AMGH, 2011.		
		\item[3] SOMMERVILLE, I. Engenharia de Software. Pearson/Prentice Hall.
	\end{itemize}

\subsection{Visão Geral do Restante do Documento}

\par
O restante deste documento inclui dois capítulos e um apêndice. O Segundo capítulo apresenta uma  descrição geral do sistema, ou seja,  uma  perspectiva funcional e objetivos do mesmo, descrição de seus usuários,   restrições e dependências  para utilização e desenvolvimento do sistema. 
\par
O Terceiro capítulo detalha os requisitos: especifica todos os requisitos funcionais e não funcionais que devem  ser implementados.
 ser implementados.


\subsection{Descrição Geral}

	\subsubsection{Perspectiva do Produto}
	\par
	O sistema consistirá em uma aplicação  desktop. A aplicação  será usada para gerenciar vendas de peças e veículos, orçamentos de diversos serviços,  controlar estoque de peças e veículos, gerir clientes e funcionários, e exibir  relatórios. As funcionalidades devem estar disponíveis em uma interface gráfica, responsável pela intermediação do funcionário com a manipulação dos dados. 
	\par
	Os dados devem ser persistidos, em um banco de dados. Isso quer dizer que o sistema será capaz de salvar dados e recuperar dados do banco de dados. Os usuários devem ter um desktop conectado  ao servidor de dados local. 
	
	
	\subsubsection{Funções do Produto}
	A seguir, detalha-se cinco das mais importantes funcionalidades do sistema. É apresentado os casos de usos de interação do ator principal, o Funcionário, e os fluxos principais e alternativos. Os demais casos de usos se encontram na seção de diagramas:
	\vspace{12px}
	
	%CDU1
	\par
	\textbf{Nome:} Autenticar Funcionário 
	\par
	\textbf{Descrição:} Autenticação dos funcionários para uso do sistema
	\par 
	\textbf{Identificador:} CDU01
	\par
	\textbf{Ator Primário:} Funcionário	
	\par
	\ldots
	\par
	\textbf{Fluxo principal}\par
	\begin{tabular}{|c|c|}
		\hline 
		Funcionário & Sistema \\ 
		\hline 
		1- Inserir cpf e senha  &  \\ 
		\hline 
		& 
		
		2 - Valida dados inseridos 
		\\ 
		\hline 
		& 
		
		3 - Exibe opções disponíveis
		\\ 
		\hline 
	\end{tabular} 
	\vspace{12px}
	\par
	\textbf{Fluxo Alternativo}(Login ou senha incorreto, funcionário inexistente ou inativado)\par
	\begin{tabular}{|c|c|}
		\hline 
		Funcionário & Sistema \\ 
		\hline 
		1- Inserir cpf e senha  &  \\ 
		\hline 
		& 
		
		2 - Valida dados inseridos 
		\\ 
		\hline 
		& 
				
		3 - Exibe mensagem de falha		
		\\ 
		\hline 
	\end{tabular} 
	\vspace{12px}

	%CDU2
	\par
	\textbf{Nome:} Cadastrar Cliente
	\par
	\textbf{Descrição:} Funcionário cadastra dados do cliente no sistema
	\par 
	\textbf{Identificador:} CDU02
	\par
	\textbf{Ator Primário:} Funcionário	
	\par
	\textbf{Precondição:} Funcionário deve estar autenticado no sistema.
	\par
	\ldots
	\par
	\textbf{Fluxo principal}\par
	\begin{tabular}{|c|c|}
		\hline 
		Funcionário & Sistema \\ 
		\hline 	
		1-Selecionar opção de cadastrar cliente &  \\ 
		\hline 
		& 
		
		2 - Valida dados inseridos 
		\\ 
		\hline 
		& 
		
		3 - Exibe opções disponíveis
		\\ 
		\hline 
		& 
		
		3 - Exibe opções disponíveis
		\\ 
		\hline 
		& 
		
		3 - Exibe opções disponíveis
		\\ 
		\hline 
		& 
		
		3 - Exibe opções disponíveis
		\\ 		
		\hline 
	\end{tabular} 
	\vspace{12px}
	\par
	\textbf{Fluxo Alternativo}(Login ou senha incorreto, funcionário inexistente ou inativado)\par
	\begin{tabular}{|c|c|}
		\hline 
		Funcionário & Sistema \\ 
		\hline 
		1- Inserir cpf e senha  &  \\ 
		\hline 
		& 
		
		2 - Valida dados inseridos 
		\\ 
		\hline 
		& 
		
		3 - Exibe mensagem de falha		
		\\ 
		\hline 
	\end{tabular} 
	\vspace{12px}

	%CDU3
	\par
	\textbf{Nome:} Autenticar Funcionário 
	\par
	\textbf{Descrição:} Autenticação dos funcionários para uso do sistema
	\par 
	\textbf{Identificador:} CDU01
	\par
	\textbf{Ator Primário:} Funcionário	
	\par
	\ldots
	\par
	\textbf{Fluxo principal}\par
	\begin{tabular}{|c|c|}
		\hline 
		Funcionário & Sistema \\ 
		\hline 
		1- Inserir cpf e senha  &  \\ 
		\hline 
		& 
		
		2 - Valida dados inseridos 
		\\ 
		\hline 
		& 
		
		3 - Exibe opções disponíveis
		\\ 
		\hline 
	\end{tabular} 
	\vspace{12px}
	\par
	\textbf{Fluxo Alternativo}(Login ou senha incorreto, funcionário inexistente ou inativado)\par
	\begin{tabular}{|c|c|}
		\hline 
		Funcionário & Sistema \\ 
		\hline 
		1- Inserir cpf e senha  &  \\ 
		\hline 
		& 
		
		2 - Valida dados inseridos 
		\\ 
		\hline 
		& 
		
		3 - Exibe mensagem de falha		
		\\ 
		\hline 
	\end{tabular} 
	\vspace{12px}

	%CDU4
	\par
	\textbf{Nome:} Autenticar Funcionário 
	\par
	\textbf{Descrição:} Autenticação dos funcionários para uso do sistema
	\par 
	\textbf{Identificador:} CDU01
	\par
	\textbf{Ator Primário:} Funcionário	
	\par
	\ldots
	\par
	\textbf{Fluxo principal}\par
	\begin{tabular}{|c|c|}
		\hline 
		Funcionário & Sistema \\ 
		\hline 
		1- Inserir cpf e senha  &  \\ 
		\hline 
		& 
		
		2 - Valida dados inseridos 
		\\ 
		\hline 
		& 
		
		3 - Exibe opções disponíveis
		\\ 
		\hline 
	\end{tabular} 
	\vspace{12px}
	\par
	\textbf{Fluxo Alternativo}(Login ou senha incorreto, funcionário inexistente ou inativado)\par
	\begin{tabular}{|c|c|}
		\hline 
		Funcionário & Sistema \\ 
		\hline 
		1- Inserir cpf e senha  &  \\ 
		\hline 
		& 
		
		2 - Valida dados inseridos 
		\\ 
		\hline 
		& 
		
		3 - Exibe mensagem de falha		
		\\ 
		\hline 
	\end{tabular} 
	\vspace{12px}

	%CDU5
	\par
	\textbf{Nome:} Autenticar Funcionário 
	\par
	\textbf{Descrição:} Autenticação dos funcionários para uso do sistema
	\par 
	\textbf{Identificador:} CDU01
	\par
	\textbf{Ator Primário:} Funcionário	
	\par
	\ldots
	\par
	\textbf{Fluxo principal}\par
	\begin{tabular}{|c|c|}
		\hline 
		Funcionário & Sistema \\ 
		\hline 
		1- Inserir cpf e senha  &  \\ 
		\hline 
		& 
		
		2 - Valida dados inseridos 
		\\ 
		\hline 
		& 
		
		3 - Exibe opções disponíveis
		\\ 
		\hline 
	\end{tabular} 
	\vspace{12px}
	\par
	\textbf{Fluxo Alternativo}(Login ou senha incorreto, funcionário inexistente ou inativado)\par
	\begin{tabular}{|c|c|}
		\hline 
		Funcionário & Sistema \\ 
		\hline 
		1- Inserir cpf e senha  &  \\ 
		\hline 
		& 
		
		2 - Valida dados inseridos 
		\\ 
		\hline 
		& 
		
		3 - Exibe mensagem de falha		
		\\ 
		\hline 
	\end{tabular} 
	\vspace{12px}
	
	\subsubsection{Características do Usuário}
	
	\begin{itemize}
	\item[] Gerente: Faz isso.
	\item[] Atendente: Faz aquilo.
	\item[] Mecânico: Faz aquilo.
	\end{itemize}

	\subsubsection{Restrições Gerais}
	\par
	O sistema deve ter conexão com o banco de dados para o funcionário se autenticar e poder utilizar os recursos do sistema. Para que isso seja possível, será necessário, no mínimo, uma conexão local com o banco de dados.
	
	\subsubsection{Suposições e Dependências}
		\begin{itemize}
			\item Ao orçar um serviço, supõem sempre que há algum funcionário mecânico disponível para fazer o serviço. A única dependência que o orçamento tem é em relação a quantidade de peças que o serviço necessita.		
		\end{itemize}
	
	
\subsection{Requisitos específicos}

\subsubsection{Requisitos Funcionais}

\par
Os requisitos RF01 ao RF29 são funcionalidades que o funcionário pode interagir com o sistema. Esses requisitos estão descritos no diagrama de Casos de Usos no Apêndice. Enquanto que os demais representam outras funcionalidades do sistema.

\begin{enumerate}[
	label=RF\arabic{*}, 
	ref=(RF\arabic{*}),
	leftmargin=1.5em,
	itemindent=4.5em]
	
%\item Inclusão de fornecedores. (Pr.: 3)\par
%O sistema deve efetuar o cadastro dos fornecedores.\par
%\item Alteração de fornecedores. (Pr.: 2)\par
%O sistema deve efetuar a alteração dos dados cadastrais de fornecedores.\par
%\item Exclusão de fornecedores. (Pr.: 1)\par
%O sistema deve efetuar a exclusão de fornecedores.\par

\item Exclusão de fornecedores. (Pr.: 1)\par
O sistema deve efetuar a exclusão de fornecedores.\par

\item Exclusão de fornecedores. (Pr.: 1)\par
O sistema deve efetuar a exclusão de fornecedores.\par

\item Exclusão de fornecedores. (Pr.: 1)\par
O sistema deve efetuar a exclusão de fornecedores.\par

\item Exclusão de fornecedores. (Pr.: 1)\par
O sistema deve efetuar a exclusão de fornecedores.\par

\item Exclusão de fornecedores. (Pr.: 1)\par
O sistema deve efetuar a exclusão de fornecedores.\par

\item Exclusão de fornecedores. (Pr.: 1)\par
O sistema deve efetuar a exclusão de fornecedores.\par

\item Exclusão de fornecedores. (Pr.: 1)\par
O sistema deve efetuar a exclusão de fornecedores.\par

\item Exclusão de fornecedores. (Pr.: 1)\par
O sistema deve efetuar a exclusão de fornecedores.\par

\item Exclusão de fornecedores. (Pr.: 1)\par
O sistema deve efetuar a exclusão de fornecedores.\par

\item Exclusão de fornecedores. (Pr.: 1)\par
O sistema deve efetuar a exclusão de fornecedores.\par

\item Exclusão de fornecedores. (Pr.: 1)\par
O sistema deve efetuar a exclusão de fornecedores.\par

\item Exclusão de fornecedores. (Pr.: 1)\par
O sistema deve efetuar a exclusão de fornecedores.\par

\item Exclusão de fornecedores. (Pr.: 1)\par
O sistema deve efetuar a exclusão de fornecedores.\par

\item Exclusão de fornecedores. (Pr.: 1)\par
O sistema deve efetuar a exclusão de fornecedores.\par

\item Exclusão de fornecedores. (Pr.: 1)\par
O sistema deve efetuar a exclusão de fornecedores.\par

\item Exclusão de fornecedores. (Pr.: 1)\par
O sistema deve efetuar a exclusão de fornecedores.\par

\item Exclusão de fornecedores. (Pr.: 1)\par
O sistema deve efetuar a exclusão de fornecedores.\par

\item Exclusão de fornecedores. (Pr.: 1)\par
O sistema deve efetuar a exclusão de fornecedores.\par

\item Exclusão de fornecedores. (Pr.: 1)\par
O sistema deve efetuar a exclusão de fornecedores.\par

\item Exclusão de fornecedores. (Pr.: 1)\par
O sistema deve efetuar a exclusão de fornecedores.\par

\item Exclusão de fornecedores. (Pr.: 1)\par
O sistema deve efetuar a exclusão de fornecedores.\par

\item Exclusão de fornecedores. (Pr.: 1)\par
O sistema deve efetuar a exclusão de fornecedores.\par

\item Exclusão de fornecedores. (Pr.: 1)\par
O sistema deve efetuar a exclusão de fornecedores.\par

\item Exclusão de fornecedores. (Pr.: 1)\par
O sistema deve efetuar a exclusão de fornecedores.\par

\item Exclusão de fornecedores. (Pr.: 1)\par
O sistema deve efetuar a exclusão de fornecedores.\par

\item Exclusão de fornecedores. (Pr.: 1)\par
O sistema deve efetuar a exclusão de fornecedores.\par

\item Exclusão de fornecedores. (Pr.: 1)\par
O sistema deve efetuar a exclusão de fornecedores.\par

\item Exclusão de fornecedores. (Pr.: 1)\par
O sistema deve efetuar a exclusão de fornecedores.\par

\item Exclusão de fornecedores. (Pr.: 1)\par
O sistema deve efetuar a exclusão de fornecedores.\par

\item Exclusão de fornecedores. (Pr.: 1)\par
O sistema deve efetuar a exclusão de fornecedores.\par

\item Exclusão de fornecedores. (Pr.: 1)\par
O sistema deve efetuar a exclusão de fornecedores.\par

\item Exclusão de fornecedores. (Pr.: 1)\par
O sistema deve efetuar a exclusão de fornecedores.\par

\item Exclusão de fornecedores. (Pr.: 1)\par
O sistema deve efetuar a exclusão de fornecedores.\par

\item Exclusão de fornecedores. (Pr.: 1)\par
O sistema deve efetuar a exclusão de fornecedores.\par

\item Exclusão de fornecedores. (Pr.: 1)\par
O sistema deve efetuar a exclusão de fornecedores.\par

\item Exclusão de fornecedores. (Pr.: 1)\par
O sistema deve efetuar a exclusão de fornecedores.\par

\item Exclusão de fornecedores. (Pr.: 1)\par
O sistema deve efetuar a exclusão de fornecedores.\par

\item Exclusão de fornecedores. (Pr.: 1)\par
O sistema deve efetuar a exclusão de fornecedores.\par

\item Exclusão de fornecedores. (Pr.: 1)\par
O sistema deve efetuar a exclusão de fornecedores.\par

\item Exclusão de fornecedores. (Pr.: 1)\par
O sistema deve efetuar a exclusão de fornecedores.\par


\end{enumerate}

\subsubsection{Requisitos Não Funcionais}



\begin{enumerate}[
	label=RNF\arabic{*}, 
	ref=(RNF\arabic{*}),
	leftmargin=1.5em,
	itemindent=4.5em]
\item (Pr.: 1): O sistema deve retornar as consultas em, no máximo, 6 segundos, em 90\% dos casos.
\item (Pr.: 1): O sistema deve retornar as consultas em, no máximo, 6 segundos, em 90\% dos casos.
\end{enumerate}

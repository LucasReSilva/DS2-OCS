\section{Levantamento de Requisitos}\label{requisitos}

\subsection{Propósito do Documento}
O objetivo deste documento é detalhar a descrição de requisitos do software OpenCarShop. Irá deixar claro o propósito do desenvolvimento do sistema bem como todas as funcionalidades, interfaces, os componentes, interações e restrições que o software contém. Este documento, inicialmente, deve ser aprovado pelos stakeholders, e assim, servir de referência para o time de desenvolvimento para desenvolver uma primeira versão do sistema.




\subsection{Escopo do Produto}
O OpenCarShop é um sistema de gestão que controlará os setores de venda de veículos, estoque, realização orçamentos de serviços, de uma concessionária de veículos de única marca.
\par 
Uma base de dados de veiculos, pecas, servicos, clientes e funcionários será feita e atualizada a medida que os usuários principais do sistemas, os funcionários, alterem e adicionem tais dados durante a gestão do sistema.
\par
Os funcionários que irão interagir com o software irão fazer a gestão através de seus desktops. O software necessita de conexão com internet para os funcionários se autenticarem no sistema e manipularem os dados. Além disso, o software irá ter uma conexão com banco de dados para que seus dados sejam armazenados de forma persistente.




\subsection{Definições e Abreviações}
	\subsubsection{Definições}
	\begin{itemize}
	\item[] Estudante: pessoa sem dinheiro.
	\end{itemize}

	\subsubsection{Abreviações}
	\begin{itemize}
	\item[] RF: Requisito Funcional.
	\item[] RNF: Requisito Não Funcional.
	\end{itemize}



\subsection{Referências}
SOMMERVILLE, I. Engenharia de Software. Pearson/Prentice Hall.



\subsection{Visão Geral do Restante do Documento}
Escrever visão geral.



\subsection{Descrição Geral}

	\subsubsection{Perspectiva do Produto}
	\ldots
	
	\subsubsection{Funções do Produto}
	\ldots
	
	\subsubsection{Características do Usuário}
	\ldots
	\begin{itemize}
	\item[] User 1: Faz isso.
	\item[] User 2: Faz aquilo.
	\end{itemize}

	\subsubsection{Restrições Gerais}
	\ldots
	
	\subsubsection{Suposições e Dependências}
	\ldots
	
	
\subsection{Requisitos específicos}

\subsubsection{Requisitos Funcionais}
\begin{enumerate}[
	label=RF\arabic{*}, 
	ref=(RF\arabic{*}),
	leftmargin=1.5em,
	itemindent=4.5em]
\item Inclusão de fornecedores. (Pr.: 3)\par
O sistema deve efetuar o cadastro dos fornecedores.\par
\item Alteração de fornecedores. (Pr.: 2)\par
O sistema deve efetuar a alteração dos dados cadastrais de fornecedores.\par
\item Exclusão de fornecedores. (Pr.: 1)\par
O sistema deve efetuar a exclusão de fornecedores.\par
\end{enumerate}

\subsubsection{Requisitos Não Funcionais}
\begin{enumerate}[
	label=RNF\arabic{*}, 
	ref=(RNF\arabic{*}),
	leftmargin=1.5em,
	itemindent=4.5em]
\item (Pr.: 1): O sistema deve retornar as consultas em, no máximo, 6 segundos, em 90\% dos casos.
\item (Pr.: 1): O sistema deve retornar as consultas em, no máximo, 6 segundos, em 90\% dos casos.
\end{enumerate}

\section{Levantamento de Requisitos}\label{requisitos}

\subsection{Propósito do Documento}
\par
O objetivo deste documento é detalhar a descrição de requisitos do software OpenCarShop, deixar claro a motivação do desenvolvimento do sistema, bem como funcionalidades, interfaces, componentes, interações e restrições que o software contém. Este documento, deve ser aprovado pelos stakeholders, e assim, servir de referência para o time de desenvolvimento, auxiliando na evolução do software.




\subsection{Escopo do Produto}
\par
O OpenCarShop é um sistema de gestão que controlará os setores de venda de veículos, estoque, realização de orçamentos de serviços de uma concessionária de veículos de única marca. 
\par
Uma base de dados de veiculos, pecas, servicos, clientes e funcionários deve ser produzida e atualizada a medida que os usuários do sistema, os funcionários, alterem e adicionem tais dados durante a utilização do sistema.
\par
Os funcionários que irão interagir com o software o farão através de seus desktops. O software necessita de conexão com o servidor de dados para que os funcionários se autentiquem no sistema e manipularem os dados. 




\subsection{Definições e Abreviações}
	\subsubsection{Definições}
	
	\begin{itemize}
		\item Funcionário: Ator principal do sistema.
		\item Orçamento: Levantamento de preços  de serviços e peças atreladas a esses serviços.
		\item Peça: Peça mecânica ou acessório veicular.
		\item Serviço: Serviço veicular, reparo, manutenção, instalação de peça.
	\end{itemize}
	
	\subsubsection{Abreviações}
	\begin{itemize}
	\item RF: Requisito Funcional.
	\item RNF: Requisito Não Funcional.
	\item CDU: Caso de Uso.
	\end{itemize}



\subsection{Referências}

	\begin{itemize}
		\item[1]  Material usado nas aulas da disciplina Desenvolvimento de Software II ministrada pelo professor Michel dos Santos Soares disponibilizado em www.sigaa.ufs.br 
		\item[2] Pressman, Roger. Engenharia de Software: Uma abordagem profissional. Porta Alegre: AMGH, 2011.		
		\item[3] SOMMERVILLE, I. Engenharia de Software. Pearson/Prentice Hall.
	\end{itemize}

\subsection{Visão Geral do Restante do Documento}

\par
O restante deste documento inclui dois capítulos e um apêndice. O Segundo capítulo apresenta uma  descrição geral do sistema, ou seja,  uma  perspectiva funcional e objetivos do mesmo, descrição de seus usuários,   restrições e dependências  para utilização e desenvolvimento do sistema. 
\par
O Terceiro capítulo detalha os requisitos: especifica todos os requisitos funcionais e não funcionais que devem  ser implementados.
 ser implementados.


\subsection{Descrição Geral}

	\subsubsection{Perspectiva do Produto}
	\par
	O sistema consistirá em uma aplicação  desktop. A aplicação  será usada para gerenciar vendas de peças e veículos, orçamentos de diversos serviços,  controlar estoque de peças e veículos, gerir clientes e funcionários, e exibir  relatórios. As funcionalidades devem estar disponíveis em uma interface gráfica, responsável pela intermediação do funcionário com a manipulação dos dados. 
	\par
	Os dados devem ser persistidos, em um banco de dados. Isso quer dizer que o sistema será capaz de salvar dados e recuperar dados do banco de dados. Os usuários devem ter um desktop conectado  ao servidor de dados local. 
	
	
	\subsubsection{Funções do Produto}
	\par
	O sistema deve gerenciar, empregados, vendas, estoque e serviços de um concessionaria. 
	
	\subsubsection{Características do Usuário}
	
	\begin{itemize}
	\item[] Gerente: Responsável pela gestão da concessionária, tem acesso as todas   funcionalidades  do sistema Open Car Shop.
	\item[] Funcionário: Responsável pelo atendimento ao cliente, geração de orçamento de serviços e vendas.
	\end{itemize}

	\subsubsection{Restrições Gerais}
	\par
	O sistema deve ter no mínimo conexão com o banco de dados para que o funcionário se autenticar e poder utilizar os recursos do sistema. 
	\par
	Somente o gerente pode realizar o cadastro, atualização  e solicitar listagem de funcionários e também realizar cadastro de fornecedor.
	\par
	Apenas funcionários com contratos ativos podem ter acesso às funcionalidades do sistema.

	
	\subsubsection{Suposições e Dependências}
		\begin{itemize}
			\item Ao gerar uma ordem  de serviço, supõe-se que sempre há algum funcionário mecânico disponível para fazer o serviço.
			\item Existe dependência que uma venda possui em relação a quantidade de peças solicitadas.			
		\end{itemize}
	
	
\subsection{Requisitos específicos}

\subsubsection{Prioridade}
	\begin{itemize}
		\item 1: Prioridade alta.
		\item 2: Prioridade media.
		\item 3: Prioridade baixa.
	\end{itemize}

\subsubsection{Requisitos Funcionais}
\par
Os requisitos listados abaixo, são funcionalidades que o funcionário pode interagir com o sistema.

\begin{enumerate}[
	label=RF\arabic{*}, 
	ref=(RF\arabic{*}),
	leftmargin=1.5em,
	itemindent=4.5em]
	
%\item Inclusão de fornecedores. (Pr.: 3)\par
%O sistema deve efetuar o cadastro dos fornecedores.\par
%\item Alteração de fornecedores. (Pr.: 2)\par
%O sistema deve efetuar a alteração dos dados cadastrais de fornecedores.\par
%\item Exclusão de fornecedores. (Pr.: 1)\par
%O sistema deve efetuar a exclusão de fornecedores.\par

\item Autenticar Funcionário (Pr.: 1):\par
Descrição: O sistema deve autenticar os funcionários, por meio de cpf e senha, de forma a não permitir acesso não autorizado.\par

\item Cadastrar Cliente (Pr.: 1):\par
O sistema deve permitir ao funcionário cadastrar clientes.\par

\item Inativar Cliente (Pr.: 3):\par
O sistema deve permitir ao funcionário inativar cadastro de clientes.\par

\item Atualizar Cliente (Pr.: 2):\par
O sistema deve permitir ao funcionário atualizar cadastro de clientes.\par

\item Listar Cliente (Pr.: 1):\par
O sistema deve listar os clientes para o funcionário.\par

\item Cadastrar Funcionário (Pr.: 1):\par
Descrição: O sistema deve permitir ao gerente cadastrar funcionários.\par

\item  Atualizar Funcionário (Pr.: 2):\par
O sistema deve permitir ao gerente atualizar os dados dos funcionários.\par

\item  Inativar Funcionário (Pr.: 3):\par
O sistema deve permitir ao gerente a inativação de funcionários..\par

\item Listar Funcionário. (Pr.: 1):\par
Descrição: O sistema deve permitir listar os funcionários pelo gerente.\par

\item Cadastrar Serviço (Pr.: 1):\par
O sistema deve permitir ao funcionário cadastrar serviços.\par

\item Atualizar Serviço (Pr.: 2):\par
O sistema deve permitir ao funcionário atualizar serviços.\par

\item Listar Serviço (Pr.: 1):\par
O sistema deve listar os Serviços para o funcionário.\par

\item Finalizar  Ordem de Serviço (Pr.: 1):\par
O sistema deve permitir ao funcionário finalizar ordens de serviços.\par

\item  Cadastrar Fornecedor (Pr.: 1):\par
O sistema deve permitir ao gerente cadastrar fornecedores.\par

\item Atualizar Fornecedor (Pr.: 2):\par
O sistema deve permitir ao gerente atualizar fornecedores.\par

\item Inativar Fornecedor (Pr.: 3):\par
O sistema deve permitir ao gerente inativar fornecedores.\par

\item Listar Fornecedor (Pr.: 1):\par
O sistema deve listar os Fornecedores para o gerente.\par

\item Cadastrar Veículo (Pr.: 1):\par
Descrição: O sistema deve permitir ao funcionário cadastrar veículos.\par

\item Listar Veículos (Pr.: 1):\par
O sistema deve permitir ao funcionário listar os veículos.\par

\item Atualizar Estoque de Veículos (Pr.: 1):\par
O sistema deve permitir ao funcionário atualizar a quantidade de itens de uma determinada peça no estoque.\par

\item Cadastrar Peça (Pr.: 1):\par
O sistema deve permitir ao funcionário cadastrar  peças.\par

\item Atualizar Peça (Pr.: 2):\par
O sistema deve permitir ao funcionário atualizar dados da peças.\par

\item  Atualizar Estoque de Peça (Pr.: 1):\par
O sistema deve permitir ao funcionário atualizar a quantidade de itens de uma determinada peça no estoque.\par

\item  Inativar Peça (Pr.: 3):\par
O sistema deve permitir ao funcionário a inativação de peças.\par

\item  Listar Peças (Pr.: 1):\par
O sistema deve permitir ao funcionário listar as peças..\par

\item Orçar serviços (Pr.: 1):\par
O sistema deve permitir ao funcionário gerar um orçamento de serviços solicitado  pelo cliente.\par

\item Vender Peça (Pr.: 1):\par
O sistema deve permitir ao funcionário realizar a venda de  itens de peça para um cliente.\par

\item Vender Veículo (Pr.: 1):\par
O sistema deve permitir ao funcionário realizar a venda de veículos para um cliente.\par

\item Autorizar Serviço (Pr.: 1):\par
O sistema deve permitir ao funcionário registrar a contratação de serviços a partir de um orçamento de serviços válido.\par

\item Pagamento de Venda de Veículos (Pr.: 1):\par
O sistema deve armazenar os pagamentos das vendas de veículos. \par

\item Pagamento de Venda de Peças (Pr.: 1):\par
O sistema deve armazenar os pagamentos das vendas de peças.\par

\item Pagamento de Contratação de Serviços (Pr.: 1):\par
O sistema deve armazenar o pagamento da contratação de serviços. .\par

\item  Gerar Comprovante de Pagamento da venda de peças  (Pr.: 2):\par
O sistema deve gerar um comprovante de pagamento pela venda de peças.\par

\item Gerar Comprovante de Pagamento da venda de veículos  (Pr.: 2):\par
O sistema deve gerar um comprovante de pagamento pela venda de veículos.\par

\item Gerar Comprovante de Pagamento de Contratação de serviço (Pr.: 1):\par
O sistema deve gerar um comprovante de pagamento pela contratação de serviços.
\par

\item Verificar disponibilidade (Pr.: 1):\par
O sistema deve verificar se a peça está disponível no estoque antes da venda.\par

\item Atualizar Estoque de peças (Pr.: 1):\par
O sistema deve atualizar a quantidade de peças após concretizar vendas.\par

\item Atualizar Estoque de veículos (Pr.: 1):\par
O sistema deve atualizar a quantidade de veículos após concretizar vendas.\par

\item Relatório de Vendas (Pr.: 1):\par
O sistema deve exibir relatório de quantidade de vendas solicitado pelo gerente.\par

\item Relatório de Cliente (Pr.: 1):\par
O sistema deve exibir histórico de compra de clientes solicitado pelo funcionário. \par


\end{enumerate}

\subsubsection{Requisitos Não Funcionais}



\begin{enumerate}[
	label=RNF\arabic{*}, 
	ref=(RNF\arabic{*}),
	leftmargin=1.5em,
	itemindent=4.5em]
	
	\item Integridade (Pr.: 1):\par
	O sistema deve permitir apenas usuários com privilégios de gerente visualizar informações de contrato dos funcionários.\par
	
	\item Tempo de Resposta (Pr.: 1):\par
	O tempo de processamento para todas as requisições devem ser de 2 segundos para  90% dos casos.  \par
	
	\item Usuários Simultâneos (Pr.: 1):\par
	O sistema deverá suportar processamento multiusuários, até 50 usuários poderão utilizar o sistema simultaneamente. \par
	
	\item Interface gráfica (Pr.: 1):\par
	Para um teste com 20 usuários, o tempo para o 90% desses usuários aprender a utilizar o sistema deve durar no máximo 3 dias.  \par
	
	\item Portabilidade (Pr.: 1):\par
	O sistema deverá ser independente de plataforma de sistema operacional. \par
\end{enumerate}

\section{Levantamento de Requisitos}\label{requisitos}

\subsection{Propósito do Documento}
\par
O objetivo deste documento é detalhar a descrição de requisitos do software OpenCarShop, deixar claro a motivação do desenvolvimento do sistema, bem como funcionalidades, interfaces, componentes, interações e restrições que o software contém. Este documento, deve ser aprovado pelos stakeholders, e assim, servir de referência para o time de desenvolvimento, auxiliando na evolução do software.





\subsection{Escopo do Produto}
\par
O OpenCarShop é um sistema de gestão que controlará os setores de venda de veículos, estoque, realização de orçamentos de serviços de uma concessionária de veículos de única marca.
\par
Uma base de dados de veículos, pecas, serviços, clientes e funcionários deve ser produzida e atualizada a medida que os usuários do sistema, os funcionários, alterem e adicionem tais dados durante a utilização do sistema.
\par
Os funcionários que irão interagir com o software o farão através de seus desktops. O software necessita de conexão com o servidor de dados para que os funcionários se autentiquem no sistema e manipularem os dados. 




\subsection{Definições e Abreviações}
	\subsubsection{Definições}
	
	\begin{itemize}
		\item Funcionário: Ator principal do sistema.
		\item Orçamento: Levantamento de preços de venda de peças e realizações de serviços.
		\item Peça: Peça mecânica ou acessório veicular.
		\item Serviço: Serviço veicular, reparo, manutenção, instalação de peça.
	\end{itemize}
	
	\subsubsection{Abreviações}
	\begin{itemize}
	\item RF: Requisito Funcional.
	\item RNF: Requisito Não Funcional.
	\item CDU: Caso de Uso.
	\end{itemize}



\subsection{Referências}

	\begin{itemize}
		\item[1]  Material usado nas aulas da disciplina Desenvolvimento de Software II ministrada pelo professor Michel dos Santos Soares disponibilizado em www.sigaa.ufs.br 
		\item[2] Pressman, Roger. Engenharia de Software: Uma abordagem profissional. Porta Alegre: AMGH, 2011.		
		\item[3] SOMMERVILLE, I. Engenharia de Software. Pearson/Prentice Hall.
	\end{itemize}

\subsection{Visão Geral do Restante do Documento}

O restante deste documento inclui três capítulos e um apêndice. O Segundo capítulo apresenta uma  descrição geral do sistema, ou seja, apresenta uma visão geral do sistema, perspectiva e objetivos do mesmo, descrição de seus usuários,   restrições e dependências  para utilização e desenvolvimento do sistema. 
O Terceiro capítulo detalha os requisitos: especifica todos os requisitos funcionais e não funcionais que devem  ser implementado.
O apêndice no final do documento apresenta os resultados obtidos no detalhamento dos requisitos.

\subsection{Descrição Geral}

	\subsubsection{Perspectiva do Produto}
	\par
	O sistema consistirá em aplicação em desktop. A aplicação em desktop será usada para gerenciar vendas de peças e veículos, orçamentos de diversos serviços,  controlar estoque de acessórios e veículos, gerir clientes e funcionários, e exibição de relatórios. As funcionalidades devem estar disponíveis em uma interface gráfica, responsável pela intermediação do funcionário com a manipulação dos dados. 
	\par
	Os dados devem persistidos, em um banco de dados. Isso dizer que o sistema será capaz de salvar dados e recuperar dados do banco de dados para as views. Por exemplo, a funcionalidade de venda deverá requisitar ao banco se a quantidade de peças é suficiente, antes de concretizar uma venda.
	\par
	Os usuários devem ter um desktop conectado a rede, local ou internet, de modo que tenha acesso ao servidor de dados, o que torna o sistema restrito a conexão com servidor remoto para funcionar.
	
	\subsubsection{Funções do Produto}
	\ldots
	
	\subsubsection{Características do Usuário}
	
	\begin{itemize}
	\item[] Gerente: Faz isso.
	\item[] Atendente: Faz aquilo.
	\item[] Mecânico: Faz aquilo.
	\end{itemize}

	\subsubsection{Restrições Gerais}
	\par
	O sistema deve ter conexão com o banco de dados para o funcionário se autenticar e poder utilizar os recursos do sistema. Para que isso seja possível, será necessário, no mínimo, uma conexão local com o banco de dados.
	
	\subsubsection{Suposições e Dependências}
		\begin{itemize}
			\item Ao orçar um serviço, supõem sempre que há algum funcionário mecânico disponível para fazer o serviço. A única dependência que o orçamento tem é em relação a quantidade de peças que o serviço necessita.		
		\end{itemize}
	
	
\subsection{Requisitos específicos}

\subsubsection{Requisitos Funcionais}

\par
Os requisitos RF01 ao RF29 são funcionalidades que o funcionário pode interagir com o sistema. Esses requisitos estão descritos no diagrama de Casos de Usos no Apêndice. Enquanto que os demais representam outras funcionalidades do sistema.

\begin{enumerate}[
	label=RF\arabic{*}, 
	ref=(RF\arabic{*}),
	leftmargin=1.5em,
	itemindent=4.5em]
	
%\item Inclusão de fornecedores. (Pr.: 3)\par
%O sistema deve efetuar o cadastro dos fornecedores.\par
%\item Alteração de fornecedores. (Pr.: 2)\par
%O sistema deve efetuar a alteração dos dados cadastrais de fornecedores.\par
%\item Exclusão de fornecedores. (Pr.: 1)\par
%O sistema deve efetuar a exclusão de fornecedores.\par

\item Exclusão de fornecedores. (Pr.: 1)\par
O sistema deve efetuar a exclusão de fornecedores.\par

\item Exclusão de fornecedores. (Pr.: 1)\par
O sistema deve efetuar a exclusão de fornecedores.\par

\item Exclusão de fornecedores. (Pr.: 1)\par
O sistema deve efetuar a exclusão de fornecedores.\par

\item Exclusão de fornecedores. (Pr.: 1)\par
O sistema deve efetuar a exclusão de fornecedores.\par

\item Exclusão de fornecedores. (Pr.: 1)\par
O sistema deve efetuar a exclusão de fornecedores.\par

\item Exclusão de fornecedores. (Pr.: 1)\par
O sistema deve efetuar a exclusão de fornecedores.\par

\item Exclusão de fornecedores. (Pr.: 1)\par
O sistema deve efetuar a exclusão de fornecedores.\par

\item Exclusão de fornecedores. (Pr.: 1)\par
O sistema deve efetuar a exclusão de fornecedores.\par

\item Exclusão de fornecedores. (Pr.: 1)\par
O sistema deve efetuar a exclusão de fornecedores.\par

\item Exclusão de fornecedores. (Pr.: 1)\par
O sistema deve efetuar a exclusão de fornecedores.\par

\item Exclusão de fornecedores. (Pr.: 1)\par
O sistema deve efetuar a exclusão de fornecedores.\par

\item Exclusão de fornecedores. (Pr.: 1)\par
O sistema deve efetuar a exclusão de fornecedores.\par

\item Exclusão de fornecedores. (Pr.: 1)\par
O sistema deve efetuar a exclusão de fornecedores.\par

\item Exclusão de fornecedores. (Pr.: 1)\par
O sistema deve efetuar a exclusão de fornecedores.\par

\item Exclusão de fornecedores. (Pr.: 1)\par
O sistema deve efetuar a exclusão de fornecedores.\par

\item Exclusão de fornecedores. (Pr.: 1)\par
O sistema deve efetuar a exclusão de fornecedores.\par

\item Exclusão de fornecedores. (Pr.: 1)\par
O sistema deve efetuar a exclusão de fornecedores.\par

\item Exclusão de fornecedores. (Pr.: 1)\par
O sistema deve efetuar a exclusão de fornecedores.\par

\item Exclusão de fornecedores. (Pr.: 1)\par
O sistema deve efetuar a exclusão de fornecedores.\par

\item Exclusão de fornecedores. (Pr.: 1)\par
O sistema deve efetuar a exclusão de fornecedores.\par

\item Exclusão de fornecedores. (Pr.: 1)\par
O sistema deve efetuar a exclusão de fornecedores.\par

\item Exclusão de fornecedores. (Pr.: 1)\par
O sistema deve efetuar a exclusão de fornecedores.\par

\item Exclusão de fornecedores. (Pr.: 1)\par
O sistema deve efetuar a exclusão de fornecedores.\par

\item Exclusão de fornecedores. (Pr.: 1)\par
O sistema deve efetuar a exclusão de fornecedores.\par

\item Exclusão de fornecedores. (Pr.: 1)\par
O sistema deve efetuar a exclusão de fornecedores.\par

\item Exclusão de fornecedores. (Pr.: 1)\par
O sistema deve efetuar a exclusão de fornecedores.\par

\item Exclusão de fornecedores. (Pr.: 1)\par
O sistema deve efetuar a exclusão de fornecedores.\par

\item Exclusão de fornecedores. (Pr.: 1)\par
O sistema deve efetuar a exclusão de fornecedores.\par

\item Exclusão de fornecedores. (Pr.: 1)\par
O sistema deve efetuar a exclusão de fornecedores.\par

\item Exclusão de fornecedores. (Pr.: 1)\par
O sistema deve efetuar a exclusão de fornecedores.\par

\item Exclusão de fornecedores. (Pr.: 1)\par
O sistema deve efetuar a exclusão de fornecedores.\par

\item Exclusão de fornecedores. (Pr.: 1)\par
O sistema deve efetuar a exclusão de fornecedores.\par

\item Exclusão de fornecedores. (Pr.: 1)\par
O sistema deve efetuar a exclusão de fornecedores.\par

\item Exclusão de fornecedores. (Pr.: 1)\par
O sistema deve efetuar a exclusão de fornecedores.\par

\item Exclusão de fornecedores. (Pr.: 1)\par
O sistema deve efetuar a exclusão de fornecedores.\par

\item Exclusão de fornecedores. (Pr.: 1)\par
O sistema deve efetuar a exclusão de fornecedores.\par

\item Exclusão de fornecedores. (Pr.: 1)\par
O sistema deve efetuar a exclusão de fornecedores.\par

\item Exclusão de fornecedores. (Pr.: 1)\par
O sistema deve efetuar a exclusão de fornecedores.\par

\item Exclusão de fornecedores. (Pr.: 1)\par
O sistema deve efetuar a exclusão de fornecedores.\par

\item Exclusão de fornecedores. (Pr.: 1)\par
O sistema deve efetuar a exclusão de fornecedores.\par


\end{enumerate}

\subsubsection{Requisitos Não Funcionais}



\begin{enumerate}[
	label=RNF\arabic{*}, 
	ref=(RNF\arabic{*}),
	leftmargin=1.5em,
	itemindent=4.5em]
\item (Pr.: 1): O sistema deve retornar as consultas em, no máximo, 6 segundos, em 90\% dos casos.
\item (Pr.: 1): O sistema deve retornar as consultas em, no máximo, 6 segundos, em 90\% dos casos.
\end{enumerate}
